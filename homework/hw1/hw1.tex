\documentclass[10pt]{amsart}
\usepackage[margin=1.5in]{geometry}
\usepackage{amssymb,amsmath,enumitem}

\DeclareMathOperator{\D}{d}
\DeclareMathOperator{\E}{e}
\newcommand{\mkvec}[1]{\begin{bmatrix} \phantom{\frac 1 2} \\ #1 \\\phantom{\frac 1 2} \end{bmatrix}}

\begin{document}

%\topmargin -1.0in
%\textheight 10.5in
\pagestyle{empty}

\newcommand{\mline}{\vspace{.2in}\hrule\vspace{.2in}}


\title{\bf { AMATH 586 Spring 2023 \\ Homework 1 ---
Due April 10 on Gradescope by 11pm} }
\maketitle
\begin{center} Be sure to do a {\tt git pull} to update your local
  version of the {\tt amath-586-2023} repository.\\  Homeworks must be
  typeset and uploaded to {\tt Gradescope} for submission.\\
  Code should be uploaded to {\tt GitHub}
  \end{center}

\mline
\begin{enumerate}[label={\bf Problem~{\arabic*}:}]
\item In this exercise you will show convergence for a discretization of
  \begin{align*}
    \begin{cases}
      -u''(x) = g(x),\\
      u'(0) = \alpha,\\
      u(1) = \beta.
    \end{cases}
  \end{align*}
  \begin{enumerate}
\item  Consider the $(m+1) \times (m+1)$ matrix
  \begin{align*}
    A = \begin{bmatrix} 1 & -1 \\ -1 & 2 &-1 \\
    & \ddots & \ddots & \ddots \\
    &&&& -1\\
    &&&-1& 2\\
  \end{bmatrix}.
    \end{align*}
    Find its Cholesky decomposition.
  \item Show that
    \begin{align}\label{unit} A^{-1} \begin{bmatrix}  1 \\ 0 \\ \vdots \\ 0 \end{bmatrix} = \begin{bmatrix}  m +1 \\ m \\ \vdots \\ 1 \end{bmatrix}.\end{align}
  \item Now show that
    \begin{align*}
      \|A^{-1}\|_1 \leq (m+1)^2, \quad \|A^{-1}\|_\infty \leq (m+1)^2.
    \end{align*}
  \end{enumerate}

\mline
  
  \item Consider the matrix (see (2.54) in LeVeque)
    \begin{align*}
      L = h^{-2} \begin{bmatrix} h & -h \\ -1 & 2 &-1 \\  
    & \ddots & \ddots & \ddots \\
    &&&& -1\\
    &&&-1& 2\\
  \end{bmatrix}.\end{align*}
\begin{enumerate}
  \item Compute $L^{-1}$ in terms of $A^{-1}$ and compute bounds for  $\|L^{-1}\|_1$ and $\|L^{-1}\|_\infty$.
  \item Explain why this is not enough to imply convergence for the one-sided approach, (2.53) in LeVeque.
    \item Use \eqref{unit} to show the method converges in both the grid 1-norm and the $\infty$-norm.
  \end{enumerate}
\mline

\item Consider $u(x) = \cos(k \pi x) \exp(-x^2)$.  Determine $g$, $\alpha$ and $\beta$ such that
  \begin{align*}
    \begin{cases} -u''(x) = g(x),\\
      u(0) = \alpha,\\
      u(1) = \beta. \end{cases}
  \end{align*}
  \begin{enumerate}
  \item Modify the code in {\tt LinearBVP.ipynb} to solve this BVP using a second-order accurate method.  Plot errors on a log-log scale for $k = 1,2,3,4$.
   \item Modify the code in {\tt LinearBVP.ipynb} to solve this BVP using a fourth-order accurate method.  Plot errors on a log-log scale for $k = 1,2,3,4$. 
  \end{enumerate}
  \mline

\item Modify the code in {\tt NonlinearBVP.ipynb} to solve
  \begin{align*}
 \begin{cases}
 w'(x) - \epsilon w'''(x) = 0,\\
 w(0) = 0,\\
 w(L) = 0,\\
 w'(L) = 1.
 \end{cases}
  \end{align*}
 Demonstrate the convergence rate by comparing the computed solution
 to the true solution for $\epsilon = 0.1, 0.01$.
  
  \end{enumerate}

\end{document}

%%% Local Variables:
%%% mode: latex
%%% TeX-master: t
%%% End:
