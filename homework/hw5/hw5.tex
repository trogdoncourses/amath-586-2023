\documentclass[10pt]{amsart}
\usepackage[margin=1.5in]{geometry}
\usepackage{amssymb,amsmath,enumitem}
\usepackage{listings}
\lstset{basicstyle=\ttfamily}

\DeclareMathOperator{\D}{d}
\DeclareMathOperator{\E}{e}
\newcommand{\half}{\frac{1}{2}}
\newcommand\unp{U^{n+1}}
\newcommand\unm{U^{n-1}}
\newcommand{\bigo}{{\mathrm O}}
\newcommand{\reals}{\mathbb R}


\newcommand{\I}{\text{i}}

\begin{document}

%\topmargin -1.0in
%\textheight 10.5in
\pagestyle{empty}

\newcommand{\mline}{\vspace{.2in}\hrule\vspace{.2in}}


\title{\bf { AMATH 586 Spring 2023 \\ Homework 5 ---
Due May 26 by 11pm} }
\maketitle
\begin{center} Be sure to do a {\tt git pull} to update your local
  version of the {\tt amath-586-2023} repository.\\  Homeworks must be
  typeset and uploaded to {\tt Gradescope} for submission.\\
  The submitted homework must include plots and descriptions of your code.\\
  Code should be uploaded to {\tt GitHub}.\\
  You must include your name and {\tt GitHub} username on your assignment.
  \end{center}

\mline
\begin{enumerate}[label={\bf Problem~{\arabic*}:}]
\item  Consider solving
  \begin{align*}
    \begin{cases} u_t + u_{xxx} = 0, \quad -1 < x < 1\\
      u(x,0) = \eta(x),\\
      u(-1,t) = u(1,t),\\
      u_x(-1,t) = u_x(1,t),\\
      u_{xx}(-1,t) = u_{xx}(1,t).\end{cases}
    \end{align*}
    This is the linear KdV (Airy) equation with periodic boundary conditions.
    \begin{itemize}
    \item Use a second-order accurate centered difference and the trapezoid method as a time-stepper. Can you see dispersive quantization? Use $\eta(x) = 1$ if $-1/2 < x < 1/2$ and $\eta(x) =0$ otherwise.
    \item Prove that the method is Lax-Richtmyer stable.  Discuss whether or not it is convergent with this initial condition.
    \end{itemize}

    \mline

  \item Consider solving
  \begin{align*}
    \begin{cases} u_t + 3 (u^2)_x + u_{xxx} = 0, \quad -L < x < L,\\
      u(x,0) = \eta(x),\\
      u(-L,t) = u(L,t),\\
      u_x(-L,t) = u_x(L,t),\\
      u_{xx}(-L,t) = u_{xx}(L,t).\end{cases}
    \end{align*}
    This is the KdV equation with periodic boundary conditions.
    \begin{itemize}
    \item Use a second-order accurate centered difference and the trapezoid method as a time-stepper to solve this problem with
      \begin{align*}
        \eta(x) = 4 \mathrm{sech}(x)^2.
      \end{align*}
      Note that you will need to implement Newton's method for this and use it at each time step.
    \item You need not prove this, but give an argument that the method Lax-Richtmyer stable.  This is easier if you use $(u^2)_x$ in the PDE as opposed to $2 u u_x$!
    \end{itemize}

    \item Consider solving the small-dispersion (semi-classical) NLS equation
  \begin{align*}
    \begin{cases}
      \I \epsilon u_t + \frac{\epsilon^2}{2} u_{xx} + |u|^2 u = 0, \quad -\infty < x < \infty,\\
      u(x,0) = A(x) e^{\I S(x)/\epsilon},\\
      A(x) = - \mathrm{sech}(x),\\
      S(x) = - \mu \log \cosh (x), \quad \mu = 0.1.
    \end{cases}
    \end{align*}
        Use the Fourier exponential integrator with Runge--Kutta 4 to solve this on $[-L, L]$ for sufficiently large $L$.  Use $\epsilon = 0.1$ and $\epsilon = 0.05$ and produce a contour plot of the squared modulus of the solution for $t \in [0,4]$ for both values of $\epsilon$.  Argue that you have chosen $L$ sufficiently large and have chosen a sufficiently large number of Fourier modes.

        \mline 

\item Consider solving
  \begin{align*}
    \begin{cases} u_t + 3 (u^2)_x + u_{xxx} = 0, \quad -L < x < L,\\
      u(x,0) = \eta(x),\\
      u(-L,t) = u(L,t),\\
      u_x(-L,t) = u_x(L,t),\\
      u_{xx}(-L,t) = u_{xx}(L,t).\end{cases}
    \end{align*}
    This is the KdV equation with periodic boundary conditions.
    \begin{itemize}
    \item Use a second-order accurate centered difference and the trapezoid method as a time-stepper to solve this problem with
      \begin{align*}
        \eta(x) = 4 \mathrm{sech}(x)^2.
      \end{align*}
      Note that you will need to implement Newton's method for this and use it at each time step.
    \item You need not prove this, give some rationale as to why you might hope this method is Lax-Richtmyer stable.  
    \end{itemize}
    \mline

    \item Consider solving the small-dispersion (semi-classical) NLS equation
  \begin{align*}
    \begin{cases}
      \I \epsilon u_t + \frac{\epsilon^2}{2} u_{xx} + |u|^2 u = 0, \quad -\infty < x < \infty,\\
      u(x,0) = A(x) e^{\I S(x)/\epsilon},\\
      A(x) = - \mathrm{sech}(x),\\
      S(x) = - \mu \log \cosh (x), \quad \mu = 0.1.
    \end{cases}
    \end{align*}
    Use the Fourier exponential integrator with Runge--Kutta 4 to solve this on $[-L, L]$ for sufficiently large $L$.  Use $\epsilon = 0.1$ and $\epsilon = 0.05$ and produce a contour plot of the squared modulus of the solution for $t \in [0,4]$ for both values of $\epsilon$.  Argue that you have chosen $L$ sufficiently large and have chosen a sufficiently large number of Fourier modes.

    \mline

    \item Consider solving the Kuramoto--Sivashinsky equation with periodic boundary conditions
 \begin{align*}
    \begin{cases} u_t + u u_x + u_{xx} + u_{xxxx} = 0, \quad -L < x <L,\\
      u(x,0) = \eta(x),\\
      u(-L,t) = u(L,t),\\
      u_x(-L,t) = u_x(L,t),\\
      u_{xx}(-L,t) = u_{xx}(L,t),\\
      u_{xxx}(-L,t) = u_{xxx}(L,t).
    \end{cases}
 \end{align*}
 Use the ETDRK4 method using Fourier series to solve this problem with $L = 16 \pi$, $\eta(x) = \cos(x/16) (1 + \sin(x/16))$.  Create a contour plot of the solution for $t \in [0,150]$.
        
  
\end{enumerate}



  
\end{document}

%%% Local Variables:
%%% mode: latex
%%% TeX-master: t
%%% End:
